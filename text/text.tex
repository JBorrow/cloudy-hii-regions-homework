\documentclass[a4paper]{article}
\usepackage{amsmath}


\title{The Physics of HII Regions}\label{the-physics-of-hii-regions}
\author{Josh Borrow}

\begin{document}

\maketitle

\section{Introduction}

In this problem, we model the H\(\textsc{ii}\) region of an O star,
within a cloud of Hydrogen and Helium. The setup used with CLOUDY for
this particular implementation is available at the bottom of this
report, but in short it contains a black body continuum that is shone
through a spherical cloud with ISM abundances.

Theoretically, one should be able to determine the edge radius of the
H\(\textsc{ii}\) region by using the Stromgren Radius, \[ R_S = \left(
\frac{3}{4\pi} \frac{S_*}{n^2 \alpha} \right)^{1/3} ~, \] where \(S_*\)
is the flux of the central star in terms of photons per unit area per
unit time, \(n\) is the number density of hydrogen in the
H\(\textsc{ii}\) cloud, and \(\alpha \approx 3\times 10^{-13}\). We can
place an `upper limit' on the number of photons that are able to ionise
the hydrogen gas by using the Stefan-Boltzmann law,
\[ S_{*, \mathrm{max}} = \frac{\sigma T^4}{E_{ion}}~, \] where
\(E_{ion}\) is the ionisation energy of neutral hydrogen (13.6 eV).

In fig.~\ref{fig:density} we show using CLOUDY how the radius of the
H\(\textsc{ii}\) region scales with the density of the cloud.

\begin{figure}
\centering
\includegraphics{../analysis/density.pdf}
\caption{The hydrogen, H\(\textsc{i}\), and H\(\textsc{ii}\) densities
are shown by the blue, orange, and green lines respectively. Note that
the different panels correspond to varying initial hydrogen density. It
is assumed that the cloud is equally dense throughout. The final panel
shows how the radius of the edge of the H\(\textsc{ii}\) and
He\(\textsc{ii}\) regions changes with cloud
density.}\label{fig:density}
\end{figure}

To determine the edge of the H\(\textsc{ii}\) region, we use the
derivative of the H\(\textsc{ii}\) abundance (see
fig.~\ref{fig:derivative} for a plot that helps describe the method).

\begin{figure}
\centering
\includegraphics{../analysis/differential.pdf}
\caption{First derivatives of the H\(\textsc{ii}\) abundance with
respect to depth into cloud. The different lines correspond to different
black body temperatures for the central star, at a fixed density of
\(n_H = 1\times 10^3\). It is clear from fig.~\ref{fig:density} that
there is a turning point in the H\(\textsc{ii}\) abundance at what
appears to be the `edge' of the cloud. By using the minima of this
derivative we can find the edge programmatically.}\label{fig:derivative}
\end{figure}

\end{document}
